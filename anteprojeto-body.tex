
\chapter{Justificativa}

A arquitetura de muitos grandes projetos de software não é documentada ou está
desatualizada \cite{SourceVersusObjectCodeExtraction}, entre as informações que
podem estar nesta documentação temos o grau de acoplamento e coesão entre os
módulos do projeto. Esta informação pode ser útil para estudar por exemplo o
impacto de possíveis alterações (rastreabilidade) no projeto
\cite{mata26-terceiro-projeto-piloto}.

Um sistema extrator de informações de dependências nos ajudaria a obter esta
informação nos dizendo por exemplo que o Módulo A depende do Módulo B e
portanto alterações no Módulo B pode trazer alguma consequencia para o Módulo
A. Existem ferramentas capaz de realizar esta tarefa, mas entre as opções
encontradas\cite{SourceVersusObjectCodeExtraction} a maioria delas faz uso do
código objeto como dados de entrada da análise.

Um problema em extrair estas informações do código objeto é que algumas
informações conhecidas pelos desenvolvedores do projeto não estão presentes no
código objeto, como por exemplo macros em projetos C/C++ que são substituidas pelo
preprocessador\cite{SourceVersusObjectCodeExtraction} em tempo de compilação.

Este problema pode ser solucionado analisando diretamente o código fonte do
projeto ao invés do código objeto, a análise feita diretamente no código fonte
traria uma informação mais próxima da arquitetura do projeto conhecida pelos
desenvolvedores.

Uma vantagem adicional em se utilizar do código fonte diretamente é a
possibilidade de analisar projetos que por algum motivo não compilem mais, seja
por dependencias antigas ou por pequenas falhas no código fonte.

\chapter{Objetivo}

Construir uma ferramenta para análide de dependência entre módulos de programas
escritos em C/C++ utilizando como base apenas o código fonte (i.e.  não dependa
de compilar o código).

A construção desta ferramenta terá como base o software
Egypt^[http://www.gson.org/egypt/], um software desenvolvido por Andreas
Gustafsson para análise de dependência de módulos em programas C/C++ que
utiliza como base o código objeto gerado pelo GCC. O Egypt é software livre e
em Janeiro de 2009 foi completamente restruturado por Antonio Terceiro o qual
tem mantido em^[http://github.com/terceiro/egypt/]. Esta versão modificada será
o alvo deste trabalho.

Para fazer a análise do código fonte pretende-se utilizar o gerador de parser
antlr^[http://www.antlr.org/] que já possui uma gramática pronta para C/C++ em
seu site.

Por fim será feito um estudo comparativo entre a ferramenta desenvolvida por
este trabalho com o extrator atual do Egypt (baseado no GCC), em termos de
informações que um consegue extrair e o outro não.

\chapter{Metodologia}

Primeiramente será feito um estudo sobre Arquitetura e Engenharia de Software,
Engenharia de Software Experimental e as ferramentas antlr e Egypt, a partir
deste estudo será implementado no Egypt um módulo, plugin ou extensão que
permita analisar e extrair informações de softwares escritos em C/C++.

Etapas
======

1. Revisar e finalizar anteprojeto com orientador.
2. Estudar Egypt, antlr e selecionar gramática para o antlr.
3. Iniciar implementação do projeto e a escrita da monografia.
4. Iniciar testes de extração de informação de dependências em projetos de softwarelivre.
5. Avaliar testes da etapa anterior e documentar resultados.
6. Revisar e finalizar monografia com orientador.

Cronograma
==========

Etapa    Fev    Mar    Abr    Mai    Jun   
------   ----   ----   ----   ----   ----   
1        xxxx
2        oxxx   ooox
3               xxxx   xxxx
4                      xxx    xxxx
5                      x      xxxx   x
6                      x      x      xxxx

Table: Etapas do projeto.
