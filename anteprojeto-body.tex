\chapter{Justificativa}

A arquiterura de software define o relacionamento entre os grandes componentes
estruturais de um projeto\cite{EngenhariaDeSoftwarePressman}. A documentação
sobre esta arquitetura é útil por exemplo para extrair métricas como coesão e
acoplamento ou para estudar o impacto de possíveis alterações (rastreabilidade)
de um projeto\cite{mata26-terceiro-projeto-piloto}.

Uma ferramente de extração de informações de dependências entre módulos é útil
para gerar esta documentação a partir do código fonte do próprio projeto.
Existem ferramentas capazes de realizar esta tarefa, mas entre as opções
pesquisadas\cite{SourceVersusObjectCodeExtraction} a maioria delas faz uso do
código objeto como dados de entrada da análise, ou seja, faz uso de dados
resultante da compilação do código fonte.

Um problema em extrair estas informações do código objeto é que algumas
informações conhecidas pelos desenvolvedores do projeto não estão presentes,
como por exemplo macros em projetos C/C++ que são substituidas pelo
preprocessador\cite{SourceVersusObjectCodeExtraction} durante a compilação.

Este problema pode ser solucionado analisando diretamente o código fonte do
projeto. Esta análise traria uma informação mais próxima da arquitetura do
projeto conhecida pelos desenvolvedores e adicionalmente possibilitaria
analisar projetos que por algum motivo não compilem mais, seja por dependências
antigas ou por pequenas falhas no código fonte.

\chapter{Objetivo}

Construir uma ferramenta para análise de dependência entre módulos de programas
escritos em C/C++ utilizando como base apenas o código fonte (i.e.  não dependa
de compilar o código).

A construção desta ferramenta terá como base o software
Egypt\footnote{http://www.gson.org/egypt}, um software desenvolvido por Andreas
Gustafsson para análise de dependência de módulos em programas C/C++ que
utiliza como base código objeto gerado pelo GCC. O Egypt é software livre e em
Janeiro de 2009 foi completamente restruturado por Antonio Terceiro o qual tem
mantido em\footnote{http://github.com/terceiro/egypt}. Esta versão modificada
será o alvo deste trabalho.

Para fazer a análise do código fonte pretende-se utilizar uma ferramenta como
antlr\footnote{http://www.antlr.org} ou Doxygen\footnote{http://www.doxygen.org}
que ofereça a capacidade de analisar projetos C/C++ sem necessidade de compilar
o código fonte.

Por fim será feito um estudo comparativo entre a ferramenta desenvolvida por
este trabalho com o extrator atual do Egypt (baseado no GCC), em termos de
informações que um consegue extrair e o outro não.

\chapter{Metodologia}

Primeiramente será feito um estudo sobre Arquitetura e Engenharia de Software,
Engenharia de Software Experimental e as ferramentas antlr, Doxygen e Egypt, a
partir deste estudo será implementado no Egypt um módulo, plugin ou extensão
que permita analisar e extrair informações de dependência entre módulos de
softwares escritos em C/C++.

\section{Etapas}

\begin{enumerate}
\item Estudar Egypt, antlr e Doxygen e escolher entre antlr ou Doxygen.
\item Iniciar implementação do projeto e a escrita da monografia.
\item Iniciar testes de extração de informação de dependências em projetos de software livre.
\item Avaliar testes da etapa anterior e documentar resultados.
\item Revisar e finalizar monografia com orientador.
\end{enumerate}

\section{Cronograma}

\begin{table}
\caption{Etapas do projeto}
\centering
\begin{tabular}{c r l r r l}
Etapa  & Fev  & Mar  & Abr  & Mai & Jun \\
\hline
1      & xxxx & x    &      &      &      \\
2      &      & xxxx & xxxx &      &      \\
3      &      &      &  xxx & xxxx &      \\
4      &      &      &    x & xxxx & x    \\
5      &      &      &    x & xxxx & xxxx \\
\hline
\end{tabular}
\end{table}
